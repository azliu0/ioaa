\documentclass[11pt]{article}

\usepackage[a4paper, margin=1in]{geometry}
\usepackage{setspace}
\onehalfspacing
\usepackage{mathtools, amssymb, amsthm, empheq, xfrac, asymptote, hyperref, graphicx, enumitem}
\usepackage[dvipsnames]{xcolor}
\usepackage{ifthen}

\hypersetup{
    colorlinks=true, 
    linktoc=all,    
    linkcolor=blue, 
    urlcolor=blue
}

% commands
\newcommand{\V}{

\vspace{\baselineskip}

}
\newcommand{\comment}[1]{\textbf{\textcolor{Red}{#1}}}
\newcommand{\image}[5][1]{
\ifnum #1=1 
    \begin{center}
        \includegraphics[width=#3cm]{#2}
    \end{center}
\else
    \begin{center}
        \includegraphics[width=#3cm]{#2}
        \includegraphics[width=#5cm]{#4}
    \end{center}
\fi
}

% \setlength{\parindent}{0cm}

\title{\huge {\fontfamily{lmss}\selectfont
\textbf{IOAA Notes}}}
% \author{{\fontfamily{cmr}\selectfont
% {Andrew Liu}}}
\author{\scshape \Large Andrew Liu}
\date{{\fontfamily{cmr}\selectfont August 2022}}

\begin{document}

\maketitle

\section{Spherical Trig}

\textit{Readings: Roy and Clarke Chapter 9, Kartunnen Chapter 2 \\(the carroll ostlie chapter is honestly not very good)}

\subsection{Triangle Formulas}

\image{images/triangle.png}{9}{}{}

Area:
\begin{align*}
    E = A + B + C - \pi \\
    \text{Area} = E\cdot r^2
\end{align*}

Spherical law of sines: 
\begin{equation*}
    \frac{\sin{a}}{\sin{A}} = \frac{\sin{b}}{\sin{B}} =\frac{\sin{c}}{\sin{C}}
\end{equation*}

Spherical law of cosines: 
\begin{equation*}
    \cos{a} = \cos{A}\sin{b}\sin{c}+\cos{b}\cos{c}
\end{equation*}
\begin{equation*}
    \cos{A} = \cos{a}\sin{B}\sin{C}-\cos{B}\cos{C}
\end{equation*}

Spherical law of cosines (alternate): 
\begin{equation*}
    \cos{B}\sin{a} = -\cos{A}\sin{b}\cos{c}+\cos{b}\sin{c}
\end{equation*}

Four parts formula: 
\begin{equation*}
    \cos{a}\cos{C} = \sin{a}\cot{b} - \sin{C}\cot{B}
\end{equation*}
\begin{equation*}
    \cos{A}\cos{c} = \sin{A}\cot{B} - \sin{c}\cot{b}
\end{equation*}

\subsection{Coordinate Systems}

\image[2]{images/rah.png}{7.5}{images/conversion.png}{7.5}
%\image{images/conversion.png}

Equatorial: (Right Ascension, Declination) = $(\alpha, \delta)$

Horizontal: (Azimuth, Altitude) = $(A, a)$

Ecliptic: (Longitude, Latitude) = $(\lambda, \beta)$

Galactic: (Longitude, Latitude) = $(\ell, b)$. 

Local Sidereal Time = $\Theta = h + \alpha$. (AKA hour angle of $\Upsilon$)

\begin{itemize}
\item Azimuth $A$ measured clockwise from either the northernmost or southernmost point on observor's horizon (usually whichever is more convenient, but stick to one notation to be consistent). 

\item RA $\alpha$ measured counterclockwise from the vernal equinox ($\Upsilon$)

\item Longitude $\lambda$ measured counterclockwise from the vernal equinox ($\Upsilon$)

\item Galactic coordinates centered at the sun. Longitude $\ell$ is measured counterclockwise from the ray $\overrightarrow{\odot GC}$, where GC is the galactic center.

\item Hour angle $h$ measured clockwise from the southern meridian. Measures the time that has elapsed after an object has culminated (in the northern hemisphere).
\end{itemize}

\subsection{Timekeeping}

\textbf{Solar vs. Sidereal:} $T$ = Local Solar Time (acronym LST not to be confused with $\Theta$ = Local Sidereal Time). $\texttt{12:00}$ local solar time corresponds to the time when the sun culminates for a particular location. On March 20th, the sun aligns with the vernal equinox and thus culminates at the same time as the vernal equinox, meaning that $T = \texttt{12:00}$ and $\Theta = \texttt{0:00}$ align on this date.

Every solar day takes $24$ hours; the time between consecutive culminations of the sun takes $24$ hours. Every sidereal day takes $23$ hours $56$ minutes $4$ seconds; for any background star, the time between consecutive culminations takes $23$ hours $56$ minutes $4$ seconds. The exact conversion is 
\begin{gather*}
    24\text{ sidereal hours} = 24\cdot \left(\frac{365.25}{366.25}\right)\text{ solar hours} \\
    24\text{ solar hours} = 24\cdot \left(\frac{366.25}{365.25}\right)\text{ sidereal hours}. 
\end{gather*}
To derive this, consider that sidereal time accounts for one additional rotation of the entire earth's orbit around the sun per year, so it necessary runs faster, i.e., a sidereal hour is less than a solar hour, and any time period measured in solar time will be numerically smaller than any period measured in sidereal time.\V

To compute solar to sidereal times at specific dates, approximate 
\[
\Theta = \begin{cases}
    T + 12\text{ hr} & \text{Vernal Equinox} \\
    T + 18\text{ hr} & \text{Summer Solstice} \\
    T + 0\text{ hr} & \text{Autumnal Equinox} \\
    T + 12\text{ hr} & \text{Winter Solstice}. 
\end{cases}
\]

Then, for example, if you want to compute sidereal time on March $31$st, this corresponds to a period of $10$ solar days that passed, which is equal to $10\cdot (366.25/365.25)$ sidereal days that passed. The difference is about $40$ minutes, as we expect, since a solar day is about $4$ minutes longer than a sidereal day (in solar time). So, on March $31$st, we have $\Theta \approx T + 12\text{ hr} + 40\text{ min}$.\V

\textbf{Zero Meridian:} Greenwich, London is the $0^{\circ}$ meridian. Greenwich Mean Time (GMT), also called Universal Time (UT) is the local solar time in Greenwich and is used as the "universal" time. Greenwich Mean Sidereal Time (GMST) is the hour angle of $\Upsilon$ ignoring the effects of nutational variation (since it is measured against the background stars). Greenwich Apparent Sidireal Time (GAST) is the actual position of $\Upsilon$. Equation of the Equinoxes: 

\begin{equation*}
    \mathcal{E}_{\Upsilon} = \text{GAST} - \text{GMST}.
\end{equation*}

Convert between Greenwich and any other location with longitude $\lambda$ with: 

\begin{equation*}
    GHA\star = HA\star + \lambda,
\end{equation*}
where west $\lambda$ is positive and east $\lambda$ is negative, so that "Longitude east, Greenwich least; Longitude west, Greenwich best".\V

\textbf{Equation of Time ($\mathcal{E}$):} The mean sun (MS) is a fictitious sun that moves along the celestial equator at the rate of $(360^{\circ})/365.25\text{ days}$. The true sun ($\odot$) moves along the ecliptic at the same rate. We introduce MS for timekeeping, since the true sun does not technically have a constant angular velocity (eccentricity of orbit) and also suffers from the $23.5^{\circ}$ obliquity. Now, equation of Time ($\mathcal{E}$):

\[\mathcal{E} = \text{RAMS} - \text{RA}\odot\]

(Note that this is backwards from the Equation of the Equinoxes). Some other notes on the equation of time that are important to remember (from that one Singapore problem): 
\begin{enumerate}
    \item The equation of time can also be written as $\mathcal{E} = \text{HA}\odot - \text{HAMS}$. 
    \item Recall that timekeeping is based on the mean sun. So, whenever the equation of time is not equal to zero, $HA\odot = 0$ does not necessarily correspond to local solar time $LST = 12^{h}$.
    \item Instead, use this: $LST = HAMS \pm 12^{h} = HA\odot - \mathcal{E} \pm 12^{h}$. This is one reason why the rising time will become earlier between the vernal equinox and summer solstice, since the equation of time becomes more and more positive (for fixed hour angle, the LST is lesser). At the summer solstice, the equation of time is equal to zero, so the only reason why the sun still rises earlier at this stage is due to the fact that the sun's declination is $23.5^{\circ}$.
\end{enumerate}


\V




\textbf{Analemma}: plot with equation of time on the horizontal axis and declination on the vertical axis. See this link for a nice visual of the projection of an Analemma: \href{https://medium.com/sentinel-hub/the-shadow-of-a-celestial-dance-90968f1f42fb}{https://medium.com/sentinel-hub/the-shadow-of-a-celestial-dance-90968f1f42fb}\V

\textbf{Time Zones:} 

\begin{center}
\begin{tabular}{|c|c|}
\hline
zone & longitude \\
\hline\hline
$\pm 0$ (UTC$\pm$0) & $0^{h}30^{m}\text{W}$—$0^{h}30^{m}\text{E}$  \\
\hline
+1 (UTC-1) & $1^{h}30^{m}\text{W}$—$0^{h}30^{m}\text{W}$  \\
\hline
+2 (UTC-2) & $2^{h}30^{m}\text{W}$—$1^{h}30^{m}\text{W}$  \\
\hline
   \vdots  &\vdots  \\
\hline
+12 (UTC-12) & $12^{h}\text{W}$—$11^{h}30^{m}\text{W}$  \\
\hline
\end{tabular}
\begin{tabular}{|c|c|}
\hline
zone & longitude \\
\hline\hline
&  \\
\hline
-1 (UTC+1) & $1^{h}30^{m}\text{E}$—$0^{h}30^{m}\text{E}$  \\
\hline
-2 (UTC+2) & $2^{h}30^{m}\text{E}$—$1^{h}30^{m}\text{E}$  \\
\hline
   \vdots  &\vdots  \\
\hline
-12 (UTC+12) & $12^{h}\text{E}$—$11^{h}30^{m}\text{E}$  \\
\hline
\end{tabular}
\end{center}

% Zone $0$ is $0^{h}30^{m}W$—$0^{h}30^{m}E$. 

% Zone $+1$ (UTC-1) is $1^{h}30^{m}W$—$0^{h}30^{m}W$. Zone $+2$ (UTC-2) is $2^{h}30^{m}W$—$1^{h}30^{m}W$ ... Zone $+12$ (UTC-12) is $12^{h}W$—$11^{h}30^{m}W$.

% Zone $-1$ (UTC+1) is $1^{h}30^{m}E$—$0^{h}30^{m}E$. Zone $-2$ (UTC+2) is $2^{h}30^{m}E$—$1^{h}30^{m}E$ ... Zone $-12$ (UTC+12) is $12^{h}E$—$11^{h}30^{m}E$.

The international date line is defined by $12^{h}\text{ E/W}$. If you're travelling from east to west (in the direction from China to America) across the international date line, then you're moving from Zone $-12$ (UTC+12) to Zone $+12$ (UTC-12). To change times from Zone $-12$ to UTC, you subtract $12$ hours; to change times from UTC to Zone $+12$, you subtract $12$ hours again. In total, you lose $24$ hours. If you're travelling from west to east, you gain $24$ hours.

\section{Cosmology}

\textit{Readings: Carroll Ostlie Chapter 29}\V

\textbf{Cosmological Principle}: the assumption that the expanding universe is both isotropic and homogenous.\V

\textbf{Hubble's Law:} 
\begin{equation*}
    v = H_0r.
\end{equation*}
$v$ is recessional velocity. $H_0$ is Hubble's constant. $r$ is distance. Works best for $r > 10$ Mpc, since smaller $r$ are obscured by \textit{peculiar velocities} (i.e. ``movement'' that is not caused by the expansion of spacetime, such as gravity between interacting galaxies). \V

\textbf{Co-moving coordinate and scale factor:}
\begin{equation*}
    r(t) = R(t)\cdot \varpi
\end{equation*}
$r(t)$ is the coordinate distance from the observer, $R(t)$ is the scale factor, and $\varpi$ is the \textit{co-moving coordinate}.

From cosmological time dilation and cosmological redshift, we have

\[R(z) = \frac{1}{1+z} = \frac{\Delta t_{emit}}{\Delta t_{obs}} = \frac{\lambda_{emit}}{\lambda_{obs}}.\]\V

\textbf{Deriving Friedmann:}

(this only works for a one-component universe of pressureless dust, so this technically isn't a fully complete proof)

\begin{center}
\begin{asy}
import graph; size(4cm); 
pen dps = linewidth(0.7) + fontsize(10); defaultpen(dps); /* default pen style */ 
pen dotstyle = black; /* point style */ 

draw(circle((0,0), 5), black);
draw(circle((0,0), 4.5), black);
fill(circle((0,0), 5), black+opacity(0.2));
fill(circle((0,0), 4.45), white);

draw((0,0));
draw((0,0)--4.5*dir(40), black, Arrow(5));

label("$r$", (0,0)--4.5*dir(40), dir(-50));
\end{asy}
\end{center}

Consider an expanding ring of dust with mass $m$ radially expanding with velocity $v$ (by the cosmological principle). Let $M_r$ be the total mass inside of the shell. Now, by conservation of energy,

\begin{equation}\label{eq:friedmannenergy}
\frac{1}{2}mv^{2}(t) - G\frac{M_{r}m}{r(t)} = -\frac{1}{2}mkc^{2}\varpi^{2}.
\end{equation}

Also, 

\[H(t) = \frac{v(t)}{r(t)} = \frac{v(t)}{R(t)\varpi} = \frac{d/dt(R(t))\varpi}{R(t)\varpi} = \frac{\dot{R}}{R}.\]

Plug stuff into Eq~\ref{eq:friedmannenergy} to eventually obtain:

\begin{equation*}
R^{2}\left(H^{2} - \frac{8}{3}\pi G\rho \right) = -kc^{2}.
\end{equation*}

This is equivalent to 

\begin{equation*}
\left(\frac{dR}{dt}\right)^{2} - \frac{8}{3}\pi G\rho R^{2} = -kc^{2},
\end{equation*}

which is also equivalent to 

\begin{equation*}
\left(\frac{dR}{dt}\right)^{2} - \frac{8}{3}\pi G\left(\frac{\rho_{m,0}}{R} + \frac{\rho_{rel,0}}{R^{2}} +\rho_{\Lambda,0}R^{2}\right) = -kc^{2}.
\end{equation*}


\textbf{Density:}

$\rho = \rho_{m} + \rho_{rel} + \rho_{\Lambda}$ accounts for matter (dark + baryonic = protons, neutrons), relativistic particles (photons and neutrinos), and dark energy. Relativistic density and dark energy density are taken as equivalent mass densities; that is, we let 
\[\rho_{rel} = \frac{u_{rel}}{c^{2}} = \frac{g^{*}aT^{4}}{2c^{2}} = \frac{2g^{*}\sigma T^{4}}{c^{3}},\]
where $u_{rel}$ is the energy density of relativistic particles, both photons and neutrinos. This differs from the normal energy density of blackbody radiation ($u=aT^{4}$, which only accounts for photons) by a factor of $g^{*}/2\approx 1.68$. $g^{*}$ itself is the ``effective number of degrees of freedom for relativistic particles''. Knowing what this means isn't super important but the $1.68$ figure has shown up before so just know that it exists. We also let 
\[\rho_{\Lambda} = \frac{\Lambda c^{2}}{8\pi G},\]
so that the Friedmann equation looks consistent ($8/3\pi G(\rho_{m}+\rho_{rel}) + 1/3\Lambda c^{2}$) when fully expanded.\V

\textbf{Other important equations:}

Fluid equation (for a pressureless universe, $P=0$ confirms $M_{r}$ constant in our simplified dust model):

\[\frac{d}{dt}(R^{3}\rho) = -\frac{P}{c^{2}}\frac{d(R^{3})}{dt}.\]

Acceleration equation:

\[\frac{d^{2}R}{dt^{2}} = -\frac{4}{3}\pi G\left(\rho + \frac{3P}{c^{2}}\right)R.\]

Equation of state:

\[P = wu = w\rho c^{2},\]

which along with the fluid equation implies

\[R^{3(1+w)}\rho_{x} = \text{constant} = \rho_{0,x},\]

where $w=0$ when $x$ is matter, $w=1/3$ when $x$ is radiation, and $w=-1$ when $x$ is dark energy. (for example, we know $P_{\text{rad}} = u_{\text{rad}}/3$).

We also have 

\[RT = T_{0},\]

which comes from Wien's law or from the relationship above for $x$ = radiation.\V

\textbf{Critical Density:}

When $k=0$ in the friedmann equation:

\begin{equation*}
\rho_{c}(t) = \frac{3H(t)^{2}}{8\pi G}.
\end{equation*}\V

\textbf{Density Paramters:}

\[\Omega_{x}(t) = \frac{\rho_{x}(t)}{\rho_{c}(t)} = \frac{8\pi G\rho_{x}(t)}{3H(t)^{2}}.\]

Try to remember approximately what the current density parameter values are:
\[\begin{cases}
\displaystyle \Omega_{m,0} =  \frac{8\pi G\rho_{m,0}}{3H_{0}^{2}} \approx 0.27 \\
\displaystyle \Omega_{rel,0} = \frac{16\pi Gg^{*}\sigma T_{0}^{4}}{3H_{0}^{2}c^{3}} \approx 8.24\cdot 10^{-5}\\
\displaystyle \Omega_{\Lambda,0} = \frac{\Lambda c^{2}}{3H_{0}^{2}} \approx 0.73.
\end{cases}
\]

(the current universe is dominated by dark energy, and relativistic particles are almost negligible so that $P/c^{2} << \rho$). $\Omega_{m,0}$ can be further broken down into the dark matter density parameter (roughly 0.22) and the baryonic matter density parameter (roughly 0.04). \V

Some very useful reformulations of the friedmann equation: 

\[H^{2}(1-\Omega)R^{2} = -kc^{2} = H_{0}^{2}(1-\Omega_{0}),\]

where $\Omega = \Omega_{m} + \Omega_{rel} + \Omega_{\Lambda}$. Also equivalent to

\[H = H_{0}(1+z)\left[(1+z)\Omega_{m,0} + (1+z)^{2}\Omega_{rel,0} + \frac{\Omega_{\Lambda,0}}{(1+z)^{2}} + 1-\Omega_{0}\right].\]\V

\textbf{Distances:}

\begin{itemize}
\item co-moving distance ($\varpi$): equal to current proper distance $d_{p,0}$. this definition is somewhat tautological but just accept it for now.
\item proper distance $(d_{p})$: distance between two events that occur simultaneously. In a flat universe, the proper distance is just equal to the coordinate distance, i.e., $d_{p} = \varpi/(1+z)$.
\item horizon distance $(d_{h})$: proper distance to the farthest observable point in the universe (the particle horizon).
\item luminosity distance $(d_{L})$: measured flux $F = L/(4\pi d_{L}^{2}) = L/(4\pi \varpi^{2} (1+z)^{2})$, so $d_{L} = \varpi (1+z)$. Oftentimes solve for luminosity distance using distance modulus (i.e. given apparent and absolute magnitude) and need to convert to proper distance *at the present time* using $d_L = \varpi (1+z) = d_{p,0} (1+z)$. 
\item angular distance $(d_{A})$: $d_{A} = D/\theta = \varpi/(1+z)$.
\end{itemize}

Example calculation of $\varpi$, $d_{p}$, and $d_{h}$ for a flat, one-component universe of pressureless dust: 

\image[2]{images/varpi-1.png}{6}{images/varpi-2.png}{6}

\section{Magnitude Stuff}

\textbf{Surface Brightness:}

SB is a measure of flux density per unit solid angle. Typically given in units of mag/arcsec$^2$ (MPSAS -- magnitudes per square arcseconds). Given SB = $\mu$, and an object with area $A$ in units of arcsec$^2$, its apparent magnitude $m$ is given by:
\[m - \mu = -2.5\log{A}.\]
Intuitively, $\mu$ is the magnitude of the object when $A = 1\text{arcsec}^2$, so the "$\log{A}$" term just represents the ratio of fluxes in an object with area $A$. Recall that flux is equal to the amount of energy (from photons) collected per unit area of a detector on earth -- the amount of energy collected is proportional to the area of the sky that the object takes up.

Surface brightness is invariant of distance! Flux density and the solid angle both drop off with distance squared ($\omega = A/r^2$).

\textbf{Terms:}

\begin{itemize}
\item \textbf{Itensity} ($I$): measured in watts per meters squared per unit steradian. This refers to \textit{total intensity}; sometimes, monochromatic intensity $I_\nu$ is used, which is in units of watts per meters squared per unit steradian per hertz. Formally, $dE = I_{\nu} dA d\nu d\omega dt$. 
\item \textbf{Flux} ($F$): (technically Flux Density) measured in watts per units squared, this is typically used to represent the amount of energy per unit time from some blackbody that passes through some surface $S$. Formally, $F = \int_S \frac{dE}{dt dA} = \int_S I\cos_{\theta}d\omega$. (the monochromatic version of this integral, i.e. having an additional factor $d\nu$ in the denominator of the first expression, also holds).
\item \textbf{Luminosity} ($L$): measured in watts. 
\item \textbf{Energy Density of Radiation} ($u$): measured in J/m$^3$.
\item \textbf{Surface Brightness} ($S$): see above.
\end{itemize}

\textbf{Formulae:}

\image{images/intensity.png}{9}{}{}

The image above depicts a monochromatic intensity (notated as $B_{\lambda}(T)$; we use the $B$ notation to denote the planck function, but $I$ for any general intensity) radiated from a blackbody with temperature $T$, passing through a sheet $dA$. Integrating over each solid angle element $d\omega$, and noting that we only care about the "vertical" perpendicular component (so multiply by $\cos\theta$), we get that, for isotropic radiation:

\begin{align*}
F &= \int_S I\cos{\theta}d\omega \\
&= I \int_{0}^{\pi/2}\int_{0}^{2\pi}\cos{\theta}\sin{\theta}\mathrm{d}\phi \mathrm{d}\theta \\
&= 2\pi I\int_{0}^{\pi / 2}1/2\cdot \sin{2\theta}\mathrm{d}\theta \\
&= \pi I.
\end{align*}

Planck functions:

\[B_{\lambda}(T) = \frac{2hc^2/\lambda^5}{e^{hc/\lambda kT} - 1}.\]
\[B_{\nu}(T) = \frac{2h\nu^3/c^2}{e^{h\nu/kT} - 1}.\]

Total intensity:
\[I = \int_0^{\infty} B_{\lambda}(T)\mathrm{d}\lambda =  \int_0^{\infty} B_{\nu}(T)\mathrm{d}\nu = F/\pi.\]

Actually performing this integral yields:

\[F = \left(\frac{2\pi^5 k^4}{15c^2 h^3}\right)T^4 = \sigma T^4.\]

Deriving energy density:
\[dE = IdAd\omega dt = I\cdot \frac{1}{c}d\omega dV \implies u = \int \frac{dE}{dV} = 4\pi/c \cdot I.\]

Useful chain of things to remember:
\[u = \frac{4\pi}{c}I =  \frac{4}{c}F = \frac{4\sigma}{c}T^4 = aT^4.\]

\section{Celestial Mechanics}

\[\mu = \frac{m_1m_2}{m_1 + m_2}\]
\[r_1 = \frac{\mu}{m_1}r, r_2 = \frac{\mu}{m_2}r, v_1 = \frac{\mu}{m_1}v, v_2 = \frac{\mu}{m_2}v.\]
(all $r$s and $v$s are vectors).

\[E = \frac{1}{2}\mu v^2 - G\frac{M\mu}{r}.\]
\[L = \mu r\times v = r\times p = \mu\sqrt{GMa(1-e^2)}.\]
\[v_p = \sqrt{\frac{GM(1+e)}{a(1-e)}}, v_a = \sqrt{\frac{GM(1-e)}{a(1+e)}}.\]
\[dA = \frac{1}{2}(r\times dr) = \frac{1}{2}(r\times vdt) = \frac{L}{2\mu}dt.\]
\[dA = \frac{1}{2}r^2d\theta.\]
\[dL/dt = (dr/dt\times p) + (r\times dp/dt) = v\times \mu v + r\times F = 0.\]

polar form of orbit equation:
\[r = \frac{p}{1 + e\cos{\theta}}.\]

Circle: $p = r$ (constant radius), $e = 0$. 
Ellipse: $p = b^2/a = a(1-e^2)$, $0 < e < 1$. \\
Parabola: $p = r_p$ (perihelion distance), $e=1$. \\
Hyperbola: $p = b^2/a = a(e^2-1)$, $e > 1$. 



\end{document}
